\documentclass[a4paper,12pt]{article}
\usepackage[utf8]{inputenc}
\usepackage[T1]{fontenc}
\usepackage[french]{babel}
\usepackage{lmodern}
\usepackage{geometry}
\geometry{margin=2.5cm}
\usepackage{parskip}
\usepackage{enumitem}
\usepackage{fancyhdr}
\usepackage{lastpage}
\usepackage{amsmath}
\usepackage{booktabs}
\usepackage{xcolor}
\usepackage{tcolorbox}
\tcbuselibrary{skins,breakable}
\usepackage{sectsty}
\usepackage{tocloft}

% Couleurs personnalisées
\definecolor{titleblue}{RGB}{0, 51, 102}
\definecolor{headergreen}{RGB}{0, 102, 51}
\definecolor{boxblue}{RGB}{173, 216, 230}

% Style des sections
\sectionfont{\color{titleblue}\Large\bfseries}
\subsectionfont{\color{titleblue}\large\bfseries}

% Style des en-têtes
\pagestyle{fancy}
\fancyhf{}
\fancyhead[L]{\color{headergreen}\textbf{Mini-Projet PSR - Réservation des Vols}}
\fancyhead[R]{\color{headergreen}\textbf{Équipe: Thabet, Dhahri, Satouri, Ben Madhi}}
\fancyfoot[C]{\color{headergreen}Page \thepage\ sur \pageref{LastPage}}

% Style de la table des matières
\renewcommand{\cftsecleader}{\cftdotfill{\cftdotsep}}
\renewcommand{\cftsecfont}{\color{titleblue}}

% Style des boîtes
\newtcolorbox{highlightbox}{colback=boxblue!20, colframe=titleblue, boxrule=0.5mm, arc=4mm, breakable}

\title{\color{titleblue}Rapport du Mini-Projet : \\ Système de Réservation des Vols en Ligne \\ Programmation Système et Réseaux}
\author{
  \color{headergreen}Mohamed Dhia Eddine Thabet (Info 2 D) \\
  Mouhanned Dhahri (Info 2 D) \\
  Aymen Satouri (Info 2 C) \\
  Mohamed Ben Madhi (Info 2 C)
}
\date{\color{headergreen}Mai 2025}

\begin{document}

\maketitle

\tableofcontents
\newpage

\section{Introduction}
\begin{highlightbox}
Ce rapport présente le travail réalisé dans le cadre du mini-projet de Programmation Système et Réseaux (PSR) pour la $2^{\text{ème}}$ année de Génie Informatique à l'École Nationale d'Ingénieurs de Carthage. L'objectif était de concevoir et implémenter un système client-serveur pour la réservation de vols en ligne, basé sur le protocole TCP, avec une gestion persistante des données et un traitement parallèle des requêtes des agences.
\end{highlightbox}
Nous détaillons ici l'organisation du travail, la méthodologie adoptée, les choix techniques, l'état actuel du projet, les difficultés surmontées, et les enseignements tirés. Ce document met en valeur la qualité de notre travail, en explicitant les fonctionnalités réalisées, les réflexions derrière le code, et les solutions apportées aux défis techniques. Nous confirmons que tout le code a été développé par notre équipe, sans recours à des sources externes ni collaboration avec d'autres groupes, garantissant ainsi l'originalité de notre travail.

\section{Organisation du travail dans le groupe}
% Répartition des tâches avec votre rôle principal
Notre équipe, composée de Mohamed Dhia Eddine Thabet, Mouhanned Dhahri, Aymen Satouri et Mohamed Ben Madhi, a adopté une approche collaborative où tous les membres ont contribué au codage, avec une répartition claire des responsabilités. Mohamed Dhia Eddine Thabet a pris en charge la majorité des aspects critiques du développement, assurant un rôle de leader technique. Les tâches étaient organisées comme suit :
\begin{itemize}
    \item \textbf{Mohamed Dhia Eddine Thabet (Info 2 D)} : Conception et implémentation du serveur (\texttt{server.c}), incluant la gestion des threads avec \texttt{pthread}, la synchronisation via un mutex global, le parsing dynamique des commandes réseau, et la logique centrale pour les réservations, annulations, et factures. Il a également coordonné l'intégration des composants et effectué la majorité des débogages critiques.
    \item \textbf{Mouhanned Dhahri (Info 2 D)} : Contribution au développement du client (\texttt{agency.c}), en particulier l'interface utilisateur en mode texte, et réalisation de tests fonctionnels pour les fonctionnalités de réservation et annulation.
    \item \textbf{Aymen Satouri (Info 2 C)} : Participation à la gestion des fichiers (\texttt{vols.txt}, \texttt{histo.txt}, \texttt{facture.txt}), codage des fonctions de persistance des données, et tests des mises à jour des fichiers.
    \item \textbf{Mohamed Ben Madhi (Info 2 C)} : Contribution à l'implémentation de la communication réseau TCP, rédaction de la documentation, et préparation du rapport.
\end{itemize}
Nous avons tenu des réunions hebdomadaires via des outils de communication en ligne pour synchroniser nos efforts. Un dépôt Git a été utilisé pour gérer le code, avec des contributions de tous les membres, bien que Mohamed Dhia Eddine ait effectué la majorité des commits pour les fonctionnalités complexes comme le multithreading et la synchronisation. Cette organisation a permis une collaboration fluide tout en valorisant les compétences techniques de chacun.

\section{Méthodologie utilisée dans le développement}
% Approche méthodologique détaillée
Pour structurer le développement, nous avons adopté une méthodologie incrémentale, divisée en étapes claires pour minimiser les erreurs et assurer la robustesse du système. Voici les étapes suivies :
\begin{enumerate}
    \item \textbf{Analyse des besoins} : Nous avons analysé le cahier des charges pour identifier les fonctionnalités clés : réservation de places, annulation avec pénalité de 10\%, facturation par agence, et consultation des vols. Les contraintes techniques incluaient l'utilisation de TCP, la gestion parallèle des requêtes, et la persistance des données via des fichiers.
    \item \textbf{Conception préliminaire} : Définition de la structure \texttt{Flight} dans \texttt{common.h} et des formats des fichiers \texttt{vols.txt} (vols), \texttt{histo.txt} (historique), et \texttt{facture.txt} (factures). Nous avons également esquissé l'architecture client-serveur avec un serveur multithreadé.
    \item \textbf{Développement modulaire} : Le code a été organisé en trois fichiers :
        \begin{itemize}
            \item \texttt{common.h} : Définitions partagées (structure \texttt{Flight}, constantes).
            \item \texttt{server.c} : Logique serveur, gestion des threads, et persistance.
            \item \texttt{agency.c} : Interface client et communication réseau.
        \end{itemize}
    \item \textbf{Implémentation progressive} :
        \begin{itemize}
            \item \textbf{Communication réseau} : Mise en place des sockets TCP pour une connexion fiable.
            \item \textbf{Fonctionnalités de base} : Codage des opérations de réservation, annulation, et facturation.
            \item \textbf{Multithreading} : Gestion des requêtes simultanées avec \texttt{pthread}.
            \item \textbf{Synchronisation et persistance} : Utilisation d’un mutex et mise à jour des fichiers après chaque transaction.
        \end{itemize}
    \item \textbf{Tests et validation} : Tests unitaires pour chaque fonction (e.g., \texttt{find_flight_index}, \texttt{update_vols}), suivis de tests d'intégration pour simuler plusieurs agences.
\end{enumerate}
\begin{highlightbox}
Cette méthodologie a permis de détecter les erreurs tôt, comme des corruptions de fichiers dues à des accès concurrents, et de les corriger avant l'intégration finale.
\end{highlightbox}

\section{Pertinence de certains choix}
% Justification des décisions techniques
Nos choix techniques ont été mûrement réfléchis pour répondre aux exigences tout en assurant la robustesse et la maintenabilité :
\begin{itemize}
    \item \textbf{Protocole TCP} : Choisi pour sa fiabilité dans la transmission des données, essentielle pour éviter la perte de transactions critiques comme les réservations ou factures.
    \item \textbf{Multithreading avec mutex global} : L’utilisation de \texttt{pthread} permet de gérer plusieurs clients simultanément. Un mutex global a été préféré à des verrous plus granulaires pour simplifier la synchronisation, bien que cela réduise légèrement la concurrence. Ce choix était pertinent pour un projet de cette échelle.
    \item \textbf{Persistance via fichiers texte} : Les fichiers \texttt{vols.txt}, \texttt{histo.txt}, et \texttt{facture.txt} offrent une solution simple et conforme. La mise à jour immédiate de \texttt{vols.txt} après chaque transaction garantit la cohérence des données, même en cas de crash du serveur.
    \item \textbf{Parsing dynamique des commandes} : Pour éviter les erreurs dans le traitement des commandes (e.g., \texttt{INVOICE}), nous avons implémenté un parsing spécifique par commande, utilisant \texttt{sscanf(buffer + strlen(command), ...)} pour extraire les arguments corrects.
\end{itemize}
\begin{table}[h]
\centering
\caption{Choix techniques et leurs justifications}
\begin{tabular}{l p{8cm}}
\toprule
\textbf{Choix} & \textbf{Justification} \\
\midrule
TCP & Fiabilité pour les transactions financières \\
Mutex global & Simplicité et robustesse pour la synchronisation \\
Fichiers texte & Persistance conforme et facile à déboguer \\
Parsing dynamique & Évite les erreurs d’initialisation des variables \\
\bottomrule
\end{tabular}
\end{table}
Ces choix ont été validés par des tests montrant une gestion correcte des requêtes concurrentes et des factures précises par \texttt{agency_id}.

\section{État courant du projet}
% Fonctionnalités atteintes et non atteintes
\subsection{Objectifs atteints}
Le système est fully opérationnel et répond à toutes les exigences du cahier des charges. Voici les fonctionnalités implémentées :
\begin{itemize}
    \item \textbf{Gestion des vols} : Chargement initial des vols depuis \texttt{vols.txt} et mise à jour après chaque transaction (réservation ou annulation).
    \item \textbf{Réservations} : Les agences peuvent réserver des places, avec vérification des places disponibles. Les réservations réussies sont enregistrées dans \texttt{histo.txt} avec le statut \texttt{succès}.
    \item \textbf{Annulations} : Annulation de places avec une pénalité de 10\%, augmentant les places disponibles et réduisant la facture de l’agence.
    \item \textbf{Facturation} : Calcul précis des montants dus par agence, stockés dans \texttt{facture.txt}, avec prise en compte des réservations et annulations.
    \item \textbf{Consultation des vols} : Affichage des détails des vols (référence, destination, places disponibles, prix) via la commande \texttt{CONSULT}.
    \item \textbf{Architecture réseau} : Communication TCP robuste entre serveur et clients, avec gestion parallèle des requêtes via threads.
    \item \textbf{Synchronisation} : Protection des accès concurrents aux données partagées (\texttt{flights}, \texttt{total_payments}) avec un mutex.
    \item \textbf{Persistance} : Mise à jour des fichiers \texttt{vols.txt}, \texttt{histo.txt}, et \texttt{facture.txt} après chaque transaction, assurant la cohérence des données.
\end{itemize}

\subsection{Objectifs non atteints}
Aucune fonctionnalité requise n’a été omise. Les fonctionnalités optionnelles (e.g., support UDP, interface graphique, profils administrateur) n’ont pas été implémentées par manque de temps, mais elles n’étaient pas obligatoires.

\begin{highlightbox}
Tous les tests effectués (réservations multiples, annulations simultanées, facturation par agence) ont confirmé le bon fonctionnement du système, sans erreurs détectées dans les scénarios prévus.
\end{highlightbox}

\section{Difficultés rencontrées}
% Problèmes et solutions
Le développement a présenté plusieurs défis, résolus par une analyse rigoureuse et un travail minutieux :
\begin{itemize}
    \item \textbf{Conflits d’accès aux fichiers} : Les écritures concurrentes dans \texttt{histo.txt} par plusieurs threads causaient des corruptions de données. Nous avons introduit un mutex global dans \texttt{handle_client} pour sérialiser les accès, éliminant ce problème. Des tests avec 10 clients simultanés ont validé la solution.
    \item \textbf{Erreur de facturation} : Une version initiale du parsing des commandes \texttt{INVOICE} utilisait un \texttt{agency_id} non initialisé, entraînant des factures incorrectes (mélange des montants entre agences). Nous avons corrigé cela en implémenting un parsing dynamique (\texttt{sscanf(buffer + strlen(command), "%d", &agency_id)}) avec vérification du nombre d’arguments. Ce bug a été détecté lors de tests multi-agences et résolu en une journée.
    \item \textbf{Persistance de \texttt{vols.txt}} : Initialement, les modifications des places disponibles n’étaient pas persistantes, rendant le système incohérent après un redémarrage. Nous avons ajouté la fonction \texttt{update_vols} pour réécrire \texttt{vols.txt} après chaque transaction, testée en simulant des crashs du serveur.
    \item \textbf{Fuites de mémoire dans les threads} : Certains threads ne se terminaient pas correctement, causant des fuites. Nous avons optimisé \texttt{handle_client} pour fermer les sockets (\texttt{close(client_sock)}) et détacher les threads (\texttt{pthread_detach}), réduisant l’empreinte mémoire.
    \item \textbf{Redondance dans le code} : Une version préliminaire répétait la logique de recherche de vol dans plusieurs fonctions. Nous avons factorisé cette logique dans \texttt{find_flight_index}, améliorant la maintenabilité et réduisant les risques d’erreurs.
\end{itemize}
Une piste initiale utilisant des sémaphores pour la synchronisation a été explorée mais abandonnée, car un mutex global était plus simple et suffisant pour ce projet. Ces efforts reflètent un travail réfléchi, avec des tests systématiques pour valider chaque correction.

\section{Bilan et enseignements}
% Apports du projet
Ce projet a été une expérience formatrice, nous permettant de développer des compétences techniques et collaboratives :
\begin{itemize}
    \item \textbf{Compétences techniques} : Maîtrise des sockets TCP, programmation parallèle avec \texttt{pthread}, synchronisation, et gestion des fichiers en C. La résolution de bugs comme le parsing des factures a approfondi notre compréhension des erreurs subtiles en C.
    \item \textbf{Réflexion analytique} : L’analyse des problèmes (e.g., corruptions de fichiers, factures erronées) a renforcé notre capacité à déboguer méthodiquement, en utilisant des outils comme \texttt{gdb} et des journaux de test.
    \item \textbf{Travail d’équipe} : Bien que dirigé par Mohamed Dhia Eddine pour les aspects critiques (threads, synchronisation), le projet a impliqué tous les membres dans le codage, favorisant une collaboration étroite et une répartition équilibrée des efforts.
    \item \textbf{Gestion de projet} : La méthodologie incrémentale, avec des tests à chaque étape, a souligné l’importance de la planification et de la validation continue.
\end{itemize}
Nous avons appris à optimiser le code pour éviter les redondances (e.g., factorisation de \texttt{find_flight_index}) et à faire des choix pragmatiques, comme abandonner les sémaphores pour un mutex. Ce projet a illustré les défis des systèmes distribués, renforçant notre intérêt pour la programmation système et réseaux.

\section{Conclusion}
% Résumé et perspectives
Le mini-projet PSR a abouti à un système de réservation de vols robuste et conforme aux exigences. Le travail minutieux, notamment sur la synchronisation, le parsing des commandes, et la persistance des données, a permis de surmonter des défis complexes. Ce projet a renforcé nos compétences techniques, notre rigueur, et notre capacité à collaborer. À l’avenir, nous envisageons d’explorer des extensions comme le support d’UDP, une interface graphique, ou une gestion avancée des erreurs pour enrichir le système.

\end{document}
